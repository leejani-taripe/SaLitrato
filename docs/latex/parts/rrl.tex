\part{Review of Related Literature}

\hspace\parindent
This segment of the project documentation provides an overview and discusses the related literature relevant to the topic of the project in which is the development of an image-based text searching application. The goal of this section is to provide a foundation for the research that will be presented in the rest of the document.


\subsection*{Full Text Search}
\hspace\parindent
A full text search is a query that looks for specific words within a large body of electronically stored text data and returns results that include all or some of those words (Full-Text Search: What Is It and How It Works | MongoDB, 2023). This is a way of searching the body of a document in much the same way that you search the web. It also requires that your documents contain actual text to index the files (Full Text Search, 2023). According to Full Text Search: How It Works (2018), full search text has a variety of uses including searching for a word or phrase in a document, searching in a web page on the internet, and searching for a name in a list. Moreover, this can be implemented by means of indexed search or by utilizing string searching algorithms like Rabin-Karp algorithm, Knuth-Morris-Pratt algorithm, simple text searching, and Boyer-Moore algorithm. Furthermore, applications like Joplin, Notepad++, Google Search, etc. contain the full text search feature (Apps with “Full-Text Search” Feature, 2023). Additionally, applications such as SeekFast and DocFetcher are some of the applications based mainly on full text search (Tran Nam Quang, 2022; SeekFast, 2023).

\subsection*{Digital Libraries}
\hspace\parindent
Access to a huge selection of digital materials, such as e-books, e-journals, digital photographs, videos, and other resources, is now made possible through the development of digital libraries. Digital libraries are websites committed to establishing and maintaining collections of electronic books and other types of content without requiring users to pay for the items they wish to go through and read (Cordón-García, et al., 2013). They offer equal access to knowledge by making it possible for consumers to access a great amount of information via internet-connected devices at any time and from any location.

\hfill

By preserving fragile or disintegrating items, digital libraries play an important part in maintaining and conserving cultural heritage, historical documents, and rare materials. They promote collaboration, citation monitoring, and data analysis while giving researchers and students access to scholarly articles, research papers, books, and other academic materials. In addition to providing access to instructional materials, works in the public domain, and user-generated content, digital libraries also extend their services to the public to promote participation and community engagement. Additionally, they offer specific features that improve user experience and speed up the process of learning, such as suggestions, bookmarking, annotation, and collaboration tools.

\subsection*{Image Processing API}
\hspace\parindent
According to Anbarjafari (2014), image processing is a process for altering an image to produce a better image or to extract some relevant information from it. It is a kind of signal processing where the input is an image, and the output can either be another image or features or characteristics related to that image. It involves performing processes on images to improve their quality, extract useful details, or enhance the way they appear. A variety of images, including photographs, medical images, satellite imagery, and more, can be processed. Image processing is a standard, modern method for manipulating photographs to obtain important information that isn’t visible in conventional images. Image processing is a popular modern strategy for modifying photographs to acquire essential knowledge that isn't obvious in regular images.

\hfill

The API enhances photos by adding detail using a variety of super-resolution methods (Abid, 2022). The API has the capacity to improve the image and scale the resolution by around four times. Modern algorithms in the API are to thank for all of this. The API can boost image quality before printing, do automatic image optimization for products, and boost social media image resolution. Developers can add image-processing features to programs by using an image-processing API. Applications may include features like image classification, OCR, image conversion, image compression, computer vision, image editing, and more using image processing APIs.

\subsection*{Pattern Searching Algorithm and Pattern Recognition}
\hspace\parindent
According to Sam (2019), algorithms for pattern searching are used to extract a pattern or substring from a larger string. There are various algorithms that exist. This kind of algorithm design's primary objective is to decrease time complexity. When searching for patterns in a larger text, the standard approach may take a long time. The Pattern Searching algorithms were included in the String algorithms and can also be referred to as String Searching Algorithms. These techniques are helpful when searching for a string within another string.

\hfill

Pattern recognition is the ability to detect arrangements of characteristics or data that yield information about a given system or data set (Wigmore, 2018). In a technological context, a pattern could be, among many other possibilities, recurrent patterns of data over time that can be utilized for anticipating patterns, particular layouts of features in images that identify objects, regular word, and phrase combinations for natural language processing (NLP), or patterns of behavior on a system that could indicate an attack. Additionally, they can classify and identify unusual parts, identify patterns and objects that are partially hidden from view and distinguish forms and objects from various perspectives.

\hfill

According to Ansari (2023), The definition of pattern recognition is the classification of data based on previously acquired knowledge or on statistical data derived from patterns and/or their representation. The potential for pattern recognition applications is one of its key features.


\subsection*{String-Matching Algorithms}
\hspace\parindent
According to GeeksforGeeks (2022), there are several known exact string-matching algorithms to use
for comparing characters. The algorithms based on character comparison are Naïve, Knuth-Morris-Pratt
(KMP), Boyer-Moore, and using the Trie data structure. In terms of the study being conducted,
Boyer-Moore Algorithm (Fig \ref{fig:Boyer-Mooore}) will be used for detecting and locating a text pattern within an image file. 

\hfill

\begin{figure}[hbt!]
   \center
    \noindent\adjustbox{width=\textwidth}{\includesvg{images/stringm}}
   \caption{Comparison between Naive and Boyer-Moore Algorithm}
   \label{fig:Boyer-Mooore}
\end{figure}

\subsection*{Boyer-Moore String-Matching Algorithm}
\hspace\parindent
According to Waruwu (2017), the Boyer-Moore (BM) algorithm is one of the well-known string-matching algorithms due to its performance in terms of matching single patterns. The algorithm compares each character in the pattern to the haystack from right to left, but the window shift remains from left to right. In addition to that, the aforementioned algorithm was proven as one of the most efficient in terms of string search applications using natural language in which it has been implemented for the functions “Search” and “Substitute” of various text editors. 

\subsection*{Comparison of Boyer-Moore Algorithm to Other String-Matching Algorithms}

\hspace\parindent
The reason for utilizing Boyer-Moore (BM) Algorithm in the study’s application software is that
several studies have demonstrated that BM Algorithm is much more efficient compared to other several
string-matching algorithms such as Brute Force, KMP, Bozorth, and Rabin and Karp in most cases
(Khumaidi et al., 2020; Layustira \& Istiono, 2021; Rasool et al., 2012; Supatmi \& Sumitra, 2019;
Dawood \& Barakat, 2020; Lin \& Soe, 2020; Buslim et al., 2020; Borah et al., 2013). According to Dawood and Barakat (2020), the overall performance evaluation of BM and KMP algorithms showed that the BM algorithm outmatched the KMP algorithm in all test scenarios. Furthermore, the 1-4 test scenarios proved that the BM algorithm is around three times faster than the KMP algorithm. The pattern size also has no significant effect on the performance of both algorithms. 

\hfill

Another study for a web-based dictionary of computer terms also concluded that BM Algorithm has a
better performance compared to the KMP algorithm. It used the Exponential Comparison Method (ECM) to
determine which is the faster and more efficient algorithm between BM and KMP. The total ECM score
shows that the BM algorithm is 0.55\% more efficient than the KMP algorithm (Khumaidi et al., 2020)

\hfill

In terms of comparison of the Boyer-Moore Algorithm to Brute-Force (BF) String-Matching algorithm,
the BM algorithm’s performance is better and much more efficient compared to BF Algorithm. According
to Lin and Soe (2020), the study results based on the number of shifts, comparisons, and runtime
concluded that the BM algorithm has better performance in comparison to the BF Algorithm. Borah et
al. (2013) also concluded that the BM algorithm is faster and more effective than the BF algorithm.
Additionally, when it comes to word suggestion search, the BM algorithm is 79.05\% more time-efficient than the BF algorithm (Layustira \& Istiono, 2021). Furthermore, Layustira and Istiono (2021) stated that the BF algorithm is preferable for string matching with shorter keywords and strings; while the BM algorithm is suited for all types and conditions of both strings and keywords being searched. 

\hfill

According to Buslim et al. (2020), the BM algorithm’s search process is more efficient than BF Algorithm, KMP Algorithm, and Rabin and Karp algorithm. This is evident in the results of the study wherein Boyer-Moore has a search speed of 0 seconds on a total of 50-word search.
